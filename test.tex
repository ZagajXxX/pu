\documentclass[12pt, letterpaper, titlepage]{article}
\usepackage[left=3.5cm, right=2.5cm, top=2.5cm, bottom=2.5cm]{geometry}
\usepackage[MeX]{polski}
\usepackage[utf8]{inputenc}
\usepackage{graphicx}
\usepackage{enumerate}
\usepackage{amsmath} %pakiet matematyczny
\usepackage{amssymb} %pakiet dodatkowych symboli
\title{Pierwszy dokument LaTeX}
\author{Dawid Zagajewski}
\date{Październik 2022}
\begin{document}
\maketitle
\section{Zainteresowania}
\subsection{- Gry}
\subsection{- Kobiety}
\subsection{- Samochody}
\subsubsection{- Szybkie}
\subsubsection{- I Wściekłe }
\newpage

\section{Składniki}
\subsection{1 kg zmielonego twarogu}
\subsection{250 g miękkiego masła}
\subsection{1 i 1/3 szklanki cukru pudru}
\subsection{6 jajek}
\subsection{1 opakowanie cukru wanilinowego}
\subsection{150 ml śmietanki 36 procent}
\subsection{4 łyżki mąki ziemniaczanej}

\textbf{Sposób Przygotowania }

\begin{enumerate}
\item Miękkie masło ubić na puszysto, stopniowo dodawać po jednym żółtku na przemian z łyżką cukru pudru, cały czas dokładnie ubijając składniki.
\item Zmniejszyć obroty miksera do średnich, dodać zmielony ser i połączyć. Teraz dodawać po kolei: cukier wanilinowy, śmietankę oraz mąkę ziemniaczaną cały czas miksując składniki na jednolitą masę. Na koniec wymieszać (delikatnie, ale dokładnie) z ubitymi na sztywno białkami.
\item Przygotować tortownicę o średnicy minimum 26 cm (mierzona od środka). Posmarować ją masłem i wysypać bułką tartą lub mielonymi migdałami lub dno wyłożyć papierem do pieczenia.
\item Masę serową wyłożyć do tortownicy i wstawić do piekarnika nagrzanego do 170 stopni C. Piec przez 60 minut. Sernik studzić stopniowo wyjmując z piekarnika (najpierw po trochu otwierając drzwiczki i lekko wysuwając sernik, w końcu wyjąć z piekarnika). Zrumieniony wierzch sernika posypać cukrem pudrem lub polać polewą czekoladową.
\end{enumerate}
\end{document}
